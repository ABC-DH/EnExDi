\documentclass[11pt,aspectratio=1610]{beamer}
\usepackage{fbb}
\usetheme{CambridgeUS}
\beamertemplatenavigationsymbolsempty
\usepackage[utf8]{inputenc}
\usepackage{amsmath,adjustbox,mathtools}
\usepackage{amsfonts}
\usepackage{hyperref}
\usepackage{graphicx}
\usepackage{xpatch}
\usepackage{makecell}
\usepackage{tabularx}
\usepackage{csquotes}
\usepackage{adjustbox}
\usepackage{pgfplots}
\usepackage{tikz}
\usetikzlibrary{shapes,calc,matrix,decorations.markings,decorations.pathreplacing,positioning, intersections,backgrounds,through,hobby}
\usepackage[english]{babel}
\AtBeginSection[]
{\begin{frame}
 \frametitle{}  
\setlength{\itemsep}{.5cm}
 \tableofcontents[currentsection,
                  hideothersubsections,
		  subsectionstyle=hide/hide/hide,
                  subsubsectionstyle=show/show/show/hide
                   ]
 \end{frame} 
 }

\AtBeginSubsection[]
{\begin{frame}
 \frametitle{}  
 \tableofcontents[currentsubsection,
	 	  currentsection,
		  sectionstyle=show/hide,
                  subsectionstyle=show/shaded/hide
                   ]
 \end{frame} 
 }



\setbeamertemplate{sections/subsections in toc}[square]
\setbeamertemplate{itemize item}[square]
\setbeamertemplate{enumerate item}[square]
\setbeamertemplate{itemize subitem}[square]

\usepackage[style=verbose,citestyle=verbose,isbn=true,url=false,
doi=true,backend=biber,maxbibnames=9, maxcitenames=4]{biblatex}
\addbibresource{biblio.bib}
\renewcommand*{\bibfont}{\tiny} 

\usepackage{tikz}
\usetikzlibrary{shapes,calc,matrix,decorations.markings,decorations.pathreplacing,positioning, intersections,backgrounds}
\tikzstyle{bag} = [align=center]

\date[June 2nd, 2025]{June 3rd, 2025}
\author[Matthias \textsc{Gille Levenson}]{\\~\\ Matthias \textsc{Gille Levenson}\\   {\scriptsize Université Aix-Marseille, École Normale Supérieure de Lyon, France}\\ {\tiny matthias [dot] gille [-] levenson [at] ens-lyon [dot] fr}\vspace{-1cm}}
\title[Introduction to TEI]{Introduction to TEI}
\titlegraphic{\hspace{0.5cm}\includegraphics[scale=0.21]{/home/mgl/Bureau/Travail/admin/logos/ensl.png}\hspace{0.5cm}\includegraphics[scale=0.23]{/home/mgl/Bureau/Travail/admin/logos/ciham.pdf}\hspace{0.5cm}}
\usepackage[justification=centering]{caption}
\begin{document}
\maketitle

\begin{frame}
\begin{itemize}
\item Those slides are freely inspired by the TEI course redacted by Ariane Pinche for the 2024 EnExDi summer school: \url{https://github.com/ABC-DH/EnExDi2024/blob/main/materials/2_OCR_TEI/slides/XML_TEI.pdf}
\end{itemize}
\end{frame}

\section{Theoric introduction}

\subsection{What is the TEI?}
\begin{frame}{What is the TEI?}
\begin{itemize}
\item A community 
\item A standard
\item A way of seeing/modelling \enquote{the} text
\end{itemize}
\end{frame}

\subsection{A little bit of history}
\begin{frame}{}
\begin{itemize}
\item Born in 1987
\item 5 versions (actual: P5)
\item Originally: SGML; switch to XML in 2007
\end{itemize}
\end{frame}

\subsection{Principles}
\begin{frame}{}
\begin{itemize}
\item Separate the appearance and the \enquote{essence} of textual objects
\end{itemize}
\begin{center}
\begin{figure}
\includegraphics[width=.8\textwidth]{img/texte.png}
\includegraphics[width=.8\textwidth]{img/apparat.png}
\caption{Fragment of an edition of the \textit{Castigos de Sancho IV}}
\end{figure}
\end{center}
\end{frame}

\begin{frame}{}
\begin{itemize}
\item Separate the appearance and the \enquote{essence} of textual objects
\end{itemize}
\begin{center}
\begin{figure}
\includegraphics[width=.8\textwidth]{img/xml_structured.png}
\caption{Its possible representation in XML-TEI}
\end{figure}
\end{center}
\end{frame}


\subsection{The TEI, what for ?}
\begin{frame}{The TEI, what for ?}
\begin{itemize}
\item Describing a text using the experience of a large community
\item Producing semantic data that can be read by the human and by the computer
\item Easing documents sharing and reusability
\end{itemize}
\end{frame}


\begin{frame}{The TEI, what for ?}
The TEI can be used for
\begin{itemize}
\item Describing a manuscript
\item Producing the edition with multiple witnesses
\item Encoding a set of letters
\item Structuring a drama
\item Describing a web-native textual object
\item ... and so on, including other media types like speech. 
\item Any kind of human communication could be theoretically represented in TEI, as long as there is some scientific interest in formating the information in this particular way
\end{itemize}
\end{frame}


\subsection{The XML format}


\begin{frame}{Conformance and validity}
\begin{itemize}
\item XML stands for \textbf{eXtensible Markup Language}.
\item It is a format that allows to describe any kind of textual (or numeric) data 
\item It is the actual format the TEI uses, but it might change/evolve in the future years.
\end{itemize}
\end{frame}

\begin{frame}{Conformance and validity}
\begin{itemize}
\item Two important concepts
\item A document \textbf{must} be XML conformant, that is, respect the rules of the XML format
\item A document \textbf{may} be validated against a schema, that is a document that verifies some rules are respected
\item Some examples of specifications: TEI, EAD, DublinCore, AltoXML, PageXML, RDF, etc...
\end{itemize}
\end{frame}


\begin{frame}{Conformance}
\begin{itemize}
\item XML is composed of elements, attributes, attribute values and text.
\end{itemize}
\begin{center}
\begin{figure}
\includegraphics[height=.5\textheight]{img/element_attribute_text.png}
\end{figure}
\end{center}
\begin{itemize}
\item There is little asumption about the types of data allowed in them in XML (hence eXtensible).
\end{itemize}
\end{frame}


\begin{frame}{One node to contain them all}
\begin{itemize}
\item One and only one top element that contains everything else
\end{itemize}
\begin{center}
\begin{figure}
\includegraphics[height=.5\textheight]{img/element_attribute_text.png}
\end{figure}
\end{center}
\end{frame}


\begin{frame}{Conformance}
\begin{itemize}
\item No overlapping elements
\end{itemize}
\begin{center}
\begin{figure}
\includegraphics[width=.9\textwidth]{img/overlapping.png}
\end{figure}
\end{center}
\end{frame}


\begin{frame}{Conformance}
\begin{itemize}
\item Some elements can contain other elements, but elements can also be empty
\end{itemize}
\begin{center}
\begin{figure}
\includegraphics[width=.9\textwidth]{img/empty.png}
\end{figure}
\end{center}
\end{frame}



\begin{frame}{Conformance}
\begin{itemize}
\item Attribute values must be inside quotes
\end{itemize}
\begin{center}
\begin{figure}
\includegraphics[height=.5\textheight]{img/attributes.png}
\end{figure}
\end{center}
\end{frame}



\begin{frame}{Conformant or not ?}
\begin{center}
\begin{figure}
\includegraphics[height=.8\textheight]{img/conformant.png}
\end{figure}
\end{center}
\end{frame}




\begin{frame}{Schemas}
\begin{itemize}
\item A schema is a document that is used to control the quality of some encoding.
\item A schema is materialized by several data formats: DTD, RNG, RNC
\item The TEI provides \textbf{rules and guidelines} (human readable), and \textbf{schemas} (machine readable) to check the validity of a given document
\item The schema represents the formalisation of your modelling of a given text or genre.
\end{itemize}
\end{frame}

\begin{frame}{Schemas}
\begin{center}
\begin{figure}
\includegraphics[height=.75\textheight]{img/validation.png}
\caption{This fragment is well-formed, but it is not TEI compliant / valid. A paragraph \texttt{p} or an anonymous block \texttt{ab} should wrap the lines.}
\end{figure}
\end{center}
\end{frame}


\begin{frame}{Schemas}
\begin{center}
\begin{figure}
\includegraphics[height=.7\textheight]{img/schemas.png}
\caption{There are multiple TEI schemas available, adapted to the user's usecases}
\end{figure}
\end{center}
\end{frame}


\begin{frame}{One Document Does it all (ODD)}
\begin{itemize}
\item ODD stands for \enquote{One Document Does it All}
\item An ODD is a TEI document that is used to create documentation and schema, with a transformation script maintained by the TEI community
\item See \cite{burnard_WhatTEIConformance_2019}
\end{itemize}
\end{frame}





\section{TEI Fundamentals}

\begin{frame}{The TEI guidelines}
\begin{itemize}
\item \url{https://tei-c.org/guidelines/}
\item The TEI guidelines is the document you will consult everyday starting now. You can print it and use it as a bedside reading.
\item The guidelines are made of two main parts: 
\begin{itemize}
\item a description of good editing practices, in natural language
\item the individual description of each element (possible attributes, elements inside, elements containing the current element, etc.)
\end{itemize}
\end{itemize}
\end{frame}

\begin{frame}{The TEI guidelines}
\begin{center}
\begin{figure}
\includegraphics[width=.9\textwidth]{img/guidelines_nl.png}
\caption{The natural language part of the TEI}
\end{figure}
\end{center}
\end{frame}


\begin{frame}{The TEI guidelines}
\begin{center}
\begin{figure}
\includegraphics[width=.8\textwidth]{img/guidelines_elements.png}
\caption{The description of the \texttt{lb} element (simplified to fit in the slide)}
\end{figure}
\end{center}
\end{frame}


\subsection{Main components}
\begin{frame}
\begin{center}
\begin{figure}
\includegraphics[height=.8\textheight]{img/tei_lite.png}
\caption{The minimal TEI document}
\end{figure}
\end{center}
\end{frame}


\begin{frame}
\begin{itemize}
\item Two main components: data and metadata 
\end{itemize}
\end{frame}


\begin{frame}{The teiHeader element}
\begin{itemize}
\item The \texttt{teiHeader} contains the metadata and al the information about the sources you are describing.
\item Four main components: \texttt{fileDesc}, \texttt{encodingDesc}, \texttt{profileDesc}, \texttt{revisionDesc}. 
% Titling and responsability
\end{itemize}
\end{frame}

\begin{frame}{The fileDesc element}
\begin{itemize}
\item The \texttt{fileDesc} contains the bibliographic information about the source. It is the only mandatory component in TEI.
\end{itemize}
\begin{center}
\begin{figure}
\includegraphics[height=.67\textheight]{img/fileDesc.png}
\caption{Example of the fileDesc taken from the TEI Guidelines}
\end{figure}
\end{center}
\end{frame}


\begin{frame}{The encodingDesc element}
\begin{itemize}
\item The \texttt{encodingDesc} is used to describe the principles that the editor has been following to produce the TEI document. 
\end{itemize}
\begin{center}
\begin{figure}
\includegraphics[height=.65\textheight]{img/encodingDesc.png}
\caption{Example of the encodingDesc taken from the TEI Guidelines}
\end{figure}
\end{center}
\end{frame}


%\begin{frame}{The profileDesc element}
%\begin{itemize}
%\item The \texttt{profileDesc} 
%\end{itemize}
%\end{frame}



\begin{frame}{The text}
\begin{itemize}
\item The \texttt{text} element contains the text \textit{per se}. It contains three main elements: \texttt{front}, \texttt{body}, \texttt{back}
\end{itemize}
\end{frame}


\begin{frame}{Facsimiles}
\begin{itemize}
\item Facsimile information is stored in a specific element after the teiHeader, in \texttt{facsimile} element. It is an element that's being used everyday more and more due to the apparition of efficient HTR algorithms.
\item This element tends to be created automatically, as it can contain lots and lots of subelements (page and line 
\end{itemize}
\end{frame}


\begin{frame}{Facsimiles}
\begin{center}
\begin{figure}
\includegraphics[width=.6\textwidth]{img/facsimile.png}
\caption{The description of the elements on the page with their coordinates}
\end{figure}
\end{center}
\end{frame}


\subsection{Structure}
\begin{frame}{Some basic structuring elements}
\url{https://www.tei-c.org/release/doc/tei-p5-doc/en/html/DS.html}
\begin{itemize}
\item \texttt{div}: for any division of a text. They can be nested, and attributes like \texttt{@type} or \texttt{@n} are used to specify the type and level of structure
\item \texttt{head}: for encoding headings
\item \texttt{p}: for encoding paragraphs
\item \texttt{ab} (for anonymous block): for encoding any sub-div block
\item \texttt{lg} (line group)
\item \texttt{l} (verse)
\end{itemize}
\end{frame}


\subsection{Document layout description and material description of sources}
\begin{frame}{Some basic structuring elements}
\url{https://tei-c.org/release/doc/tei-p5-doc/en/html/MS.html}
\begin{itemize}
\item \texttt{pb}: page beginning
\item \texttt{cb}: columns beginning
\item \texttt{lb}: line begining
\item \texttt{fw} (forme work): for encoding headers, footers, page number, catchwords, etc
\end{itemize}
\end{frame}


\begin{frame}{Some basic structuring elements}
\begin{center}
\begin{figure}
\includegraphics[height=.65\textheight]{img/material_description.png}
\end{figure}
\end{center}
\end{frame}

\subsection{Exercise 1. Encoding a poem.}
\begin{frame}{Shakespeare's Sonnet 18}
\begin{itemize}
\item Encode the Sonnet 18, available in the github repo: \texttt{materials/2\_OCR\_TEI/TEI/data/sonnet\_18.txt}
\item The \enquote{verse} section of the guidelines will help you here:   \url{https://www.tei-c.org/release/doc/tei-p5-doc/en/html/VE.html}
\item The type of each stanza has to be specified, as well as its position in the poem when it is relevant
\end{itemize}
\end{frame}



\subsection{Editing documents}
\begin{frame}{Editing documents}
\begin{itemize}
\item \url{https://www.tei-c.org/release/doc/tei-p5-doc/en/html/TC.html}
\item \texttt{app}: the apparatus entry
\item \texttt{lem}: the accepted reading, that will be shown in the edition
\item \texttt{rdg}: the rejected reading(s), that might be indicated in the apparatus 
\end{itemize}
\begin{center}
\begin{figure}
\includegraphics[width=.6\textwidth]{img/tei_app.png}
\caption{Example of apparatus modelling (taken from  the TEI Guidelines)}
\end{figure}
\end{center}
\end{frame}


\begin{frame}{Editing documents}
\begin{itemize}
\item Each witness, manuscript or print, has to be specified with an attribute, \texttt{@wit}
\item The witnesses has to be described somewhere, in general in the teiHeader and more precisely in the sourceDesc: see the usage of the listWit in the guideline.
\end{itemize}
\end{frame}

\begin{frame}{Exercise 2}
\begin{itemize}
\item Represent the edition by Delphine Demelas of the Chanson d'Otinel in XML-TEI. Try to preserve any information you can.
\end{itemize}
\end{frame}
%\subsection{Indexes}
%\begin{frame}{Indexing people and places}
%\begin{itemize}
%\item Indexing is useful for analysis purposes
%\item It allows to point any reference to an object (person, place, other) in the text
%\item The list of entities will be created in the teiHeader
%\end{itemize}
%\end{frame}




%\subsection{Exercise 2. Encoding a complex novel.}
%\begin{frame}{Alain Damasio, \textit{La Horde du Contrevent}}
%\begin{itemize}
%\item Encode three page of the novel \enquote{La Horde du Contrevent} by Alain Damasio
%\end{itemize}
%\end{frame}




\section{Conclusion}
\begin{frame}{}
\begin{itemize}
\item TEI-based editions without publication of the XML sources is quite unuseful
\item Please publish your data alongside its documentation !
\end{itemize}
\end{frame}

\subsection{What's next ? Manipulating XML trees}
\begin{frame}{XPath}
\begin{itemize}
\item XPath is the base language to navigate trees
\end{itemize}
\begin{center}
\begin{figure}
\includegraphics[width=.95\textwidth]{img/xpath.png}
\caption{A (kind of) simple XPath query, inside quotes}
\end{figure}
\end{center}
\end{frame}


\begin{frame}{XSLT}
\begin{itemize}
\item XSLT is a transformation language that is built on the same logic as XML: nesting
\item It is usefull for creating complex (web-based or \LaTeX) editions
\end{itemize}
\begin{center}
\begin{figure}
\includegraphics[width=.7\textwidth]{img/xslt_example.png}
\caption{A rule to tokenize a text into words and punctuation with XSLT}
\end{figure}
\end{center}
\end{frame}

\begin{frame}{XQuery}
\begin{itemize}
\item XQuery is a query language that is used to build XML databases and create dynamic editions.
\end{itemize}
\begin{center}
\begin{figure}
\includegraphics[width=.7\textwidth]{img/xquery.png}
\caption{Example of query on a collection of tombstones inscriptions encoded in XML-TEI. Taken from \href{https://tei-c.org/Vault/Talks/OUCS/2006-02/exercise-xquery.pdf}{James Cummings' workshop} at TEI Conference 2006.}
\end{figure}
\end{center}
\end{frame}



\begin{frame}{Python}
\begin{itemize}
\item Python is usefull to plug external tools for text processing. It is really helpfull for NLP tasks (annotation, segmentation, etc)
\item It can be more performant for treating large corpora than XSLT/XQuery
\item It follows a linear logic and therefore is not adapted to in-depth transformations of XML sources (not suited for editing for instance)
\end{itemize}
\begin{center}
\begin{figure}
\includegraphics[width=.7\textwidth]{img/python_example.png}
\caption{Some code that extracts each line of a transcription to detect hyphenated lines}
\end{figure}
\end{center}
\end{frame}



\begin{frame}[allowframebreaks]
\nocite{*}
\printbibliography
\end{frame}

\end{document}
